
\documentclass[a4paper, 11pt]{article}

\usepackage[margin=2cm]{geometry}

\usepackage{amsfonts, amsmath, amssymb, amsthm}
%\usepackage{graphicx}
\usepackage{tikz}
\usepackage{hyperref}
\usepackage[caption=false]{subfig}

\numberwithin{equation}{section}

\usepackage[noend]{algpseudocode}
\usepackage{algorithm}

\newcommand{\alphabet}[1]{\mathcal{#1}}

\renewcommand{\thesection}{\arabic{section}}

% Change date format to use it for version number
\usepackage[yyyymmdd]{datetime}
\renewcommand{\dateseparator}{.}


\title{Data Compression - Report}
\date{Version \today}
\author{lrfk99}

\begin{document}

\maketitle


\section{Lempel-Ziv-Welch (LZW)}

Lempel-Ziv-Welch (LZW) is appropriate for a large tex file since there are likely to be repeated occurrences 
of both sequences of letters (words) and sequences of words (sentences). 
This is where LZW excels, since it builds up longer and longer sequences of characters in its dictionary 
which can then be encoded as a vastly shorter sequence, possibly just 2 bytes. 
My idea for how to use LZW was to encode each token as the smallest possible number of bytes, 
each byte can encode a number between 0 and 255, this means with 2 bytes we can store any number between 
0 and 65535. 
Therefore if our maximum token is 65535, we can encode every token as 2 bytes, supporting the worst case 
scenario for a file of size 65KB. 
However if the file is large enough that we require more than 65535 dictionary entries, we can increase 
the number of bytes used to encode each token to 3 (supporting 16MB files) or to 4, accommodating the worst 
case scenario for files up to 4GB. 
The change is made by switching the mode used to encode the tokens from short unsigned integer (H) to long 
unsigned integer (L). 
Since the tex file that my program will be tested on is around 200 pages, which should be around 1MB in size, 
it is most likely that the 3 byte method will be used, but the upper limit of 4 bytes is clearly sufficient. 
The number of bytes required to encode each token is included in the encoded file (as the first byte), 
which means the decoder knows how to decode the tokens and build the dictionary properly. 
The program performs well on the lecture notes files (1.3 ratio) and achieves good results on many large 
text benchmarks, scoring a ratio of 2.46 on the Bible, 1.99 on As You Like It, 2.17 on Alice in Wonderland, 
and 2.28 on The Project Gutenberg World Factbook. 
I chose to use LZW over LZ variants such as LZ77 and ROLZ since I think that for this assignment 
LZW is most appropriate. 
There may be huge gaps between repeated instances of sequences in the large tex file, which would hurt 
the efficiency of LZ77, and ROLZ is not necessary here since we only have to compress one tex file, 
so there is no context variation. 


\section{Prediction by Partial Matching (PPM)}

Prediction by Partial Matching (PPM) is appropriate for a large English tex file since it is possible to 
predict the next character given the previous character(s), meaning that a statistical model of English 
text is appropriate to use. 
This statistical model can be 'trained' on lots of English text compression benchmarks 
(Alice in Wonderland, Bible, Factbook) and other tex documents such as the lecture notes for this module. 
It works by using a statistical model, stored as an external json file to be used by both the encoder and decoder, 
which denotes how likely a given character is to appear next, given the context of the previous $n$ characters, 
where $n$ is the length of the context. 
The encoder uses an adaptive arithmetic encoder to encode sequences from the input stream with a certain 
probability, and the uses this model to encode each byte of input data from the tex file as a shorter 
sequence of bits. 
The goal of the encoder is to use the shortest possible sequence of bits for each input sequence. 
This is how compression is achieved. 
In my implementation I take the entire bit stream from the encoder and convert it to bytes by splitting 
every 8 bits up and writing the byte representation to the lz file. 
The decoder then reads the binary file as input and converts it back to a stream of bits. 
Using the same external json file it can decode this bit stream back into the original tex file, since it 
follows the same algorithm as the encoder but in reverse. 
With regards to using Method A, B, or C to assign frequencies to the escape character, I chose to use Method C. 
This is because it is the most suitable for this implementation since it takes into account the fact that 
some contexts can be followed by virtually any other character by giving the escape symbol an appropriate count, 
whilst not reducing the count of other symbols. 
I chose to use a maximum context length of 4 due to memory constraints, since increasing this parameter increases 
memory usage exponentially at $O(257^{n})$. 
Decreasing this parameter would ease memory usage and make the program run much faster, however it would produce 
significantly worse compression results. 
PPM performs very well on text compression benchmarks such as enwik8 and enwik9 \cite{TextBenchmark}. 
My attempt at implementing PPM was unsuccessful within the time constraints, so I have left parts of my code in 
the encoder file, commented out, as a demonstration of my efforts. 


\section{Burrows-Wheeler Transform (BWT)}

The Burrows-Wheeler Transform (BWT) rearranges a character string into runs of similar characters \cite{burrows1994block}. 
The BWT is appropriate for this assignment because it is possible to scan and manipulate the entire file 
before having to encode it. 
This is a fundamental difference between BWT and PPM, since PPM does not require access to the entire file 
at once. 
BWT works by taking an input string with little redundancy or predictability and outputting a string with 
a large amount of repeating characters and thus high redundancy. 
The crucial reason this works is that the 'sorting' of the characters done by Burrows-Wheeler Forward 
in the encoding process can be reversed by Burrows-Wheeler Back in the decoding process. 
To transform a string in the forward section of the algorithm, we insert a marker character at the start 
of the string, and then generate all of the cyclic shifts of this new string. 
Next, we sort these cyclic shifts in ascending order and take the last character of each shifted string. 
This is the output, it's a partially sorted version of our original string, with a marker character added 
at some index. 
The important property of this partially sorted string is that it has more redundancy than the original, 
meaning it lends itself much better to compression algorithms. 
The BWT is a pre-processing step which is done before using a compression algorithm such as LZW or 
even just run-length encoding. 
This algorithm improves the efficiency of the compression algorithm used afterwards, without storing 
any extra data other than the position of the first original character. 
This means that BWT is a virtually free way to get better compression results since it only increases 
computation requirements. 
Many of the best text-based compression algorithms use BWT \cite{TextBenchmark}. 
I did not end up implementing BWT because I was unable to produce a working version of the transform 
within the time constraints. 
Had I been able to implement the transform as a preprocessing step before the LZW or PPM compression, I expect 
it would have improved the compression ratio significantly, with the only trade-off being additional computation time. 


\section{Move-to-front transform (MTF)}

The move-to-front (MTF) transform, which I first read about in \cite{GoogleCompression}, 
is a very powerful transform which significantly improves the compression performance of 
various algorithms such as Huffman coding and LZW. 
It works by moving the last symbol seen to the front, such that its index in the table becomes 0. 
This is similar to what the Run-Length-Transform (RLT) does, however, the MTF transform 
is a big improvement over RLT since it transforms not just runs of bytes but also other kinds of 
string patterns. 
This is possible since it actually outputs a character's position in the symbol table, rather than the 
character itself, allowing it to output the same sequence of position codes but this sequence might denote 
different strings at different states of the process. 
This idea of the same token representing different strings at different times is possible because 
the symbol table is transformed in a defined way at both the encoding and decoded stages, allowing 
the decoder to recreate the table exactly as it was when the encoder reached a given character. 
While this transform works well on highly-redundant files such as images, 
it does not perform well on user-created text-based files since they are often not highly-redundant 
and don't contain many runs of distinct bytes and similar-context strings. 
For the purposes of this assignment I need to use a transform which can accept seemingly-random input 
and produce an output of redundant bytes, and this can be done much more effectively by the BWT. 
Implementing MTF rather than BWT would produce worse compression results, so I will not use the MTF transform here. 


\section{Context Mixing (CM)}

Context Mixing is based on PPM in that it is comprised of a predictor and a coder, but it is an improvement 
over PPM in that the predictor is actually a combination of multiple models which have been trained on 
different contexts. 
The most useful contexts for compressing tex files are the previous $n$ bytes before the symbol, 
(the context used in PPM), and the previous $n$ words. 
There are other contexts which can be used for different file types to achieve better results. 
The weighted combination of these two models often provides a more accurate prediction than either 
model on its own. 
The disadvantage of using Context Mixing is that it takes longer and uses more memory, since we have to 
store two models and read both of them to calculate our more accurate prediction. 
This is a trade-off worth making in this assignment since time taken and memory usage are not assessed, but 
compression ratio is, and combining statistical models will yield more accurate predictions of the next 
character and thus better compression results. 
Context Mixing produces the best lossless text compression results outside of using a neural network \cite{TextBenchmark}, 
and is also used in the PAQ series of lossless data compression programs \cite{TextCompression}. 
Within the time constraints I was unable to produce a working Context Mixing encoder and decoder. 
However if I had been able to create a CM system, I anticipate that it would have produced the second 
best compression ratio of these 6 ideas, second only to LSTM. 


\section{Long short-term memory (LSTM)}

Long short-term memory (LSTM) is appropriate for this assignment since it achieves state-of-the-art 
results on numerous text compression benchmarks \cite{cmix,TextBenchmark}. 
LSTM is similar to context mixing in that it uses a combination of multiple models to predict the next character. 
The difference is that these models are built using recurrent neural networks, instead of 
simply the previous $n$ words or $n$ bytes, providing more accurate predictions and better compression results. 
LSTM is used in cmix \cite{cmix}, which produces the best results for enwik8 and up until 2021 \cite{bellard2021lossless} 
produced the best compression on enwik9 \cite{TextBenchmark}. 
I did not implement LSTM since I do not have sufficient experience with recurrent neural networks and within 
the time constraints I was unable to familiarise myself with the topic enough to produce a working system. 
If I had implemented LSTM, I expect it would have produced the best results of any of these 6 techniques 
on the large tex file. 


\bibliographystyle{plain}
\bibliography{references}


\end{document}
